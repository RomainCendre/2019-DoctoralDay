\section{Introduction}
\label{sec:intro}
Skin cancer is the most encountered form of human malignancy, with an increasing incidence rate over the years. This pathology can have a heavy social impact on those affected, not only decreasing their quality of life, but also potentially becoming lethal. In addition, it has significant economic consequences, with an estimated cost of 8 billion dollars per year in the United States~\cite{Farberg2017a}. However, most of these repercussions could be avoided with an early detection and appropriate surgeries~\cite{Farberg2017a}.\par
Until these days, biopsies are the gold standard to check skin pathology. However, they still remain time consuming, invasive and inconvenient for experts and patients. Consequently, several imaging techniques were developed to perform an early detection of these diseases, some of which are of common use by experts. For instance, Clinical Photography and Dermatoscopy are both examples of affordable, easy to use techniques largely exploited by dermatologist. Dermatoscopy tends to replace Clinical Photography as it significantly improves the quality of diagnosis made by experts, thanks to acquisition of homogeneous images~\cite{Sinz2017}.\par
Many research papers based on automatic classification for dermatology focus on dermatoscopy. Most of them obtain acceptable results on Melanocytic Skin Cancer pathologies~\cite{Iyatomi2010}. The oldest methods expect to find the most pertinent combination of preprocessed and hand-crafted features, to be used in a machine learning scheme~\cite{Rastgoo2015}~\cite{Pathan2018}. Recently, complex approaches based on Deep Learning techniques appear showing impressive results~\cite{Esteva2017}.\par
\ac{rcm} is another type of imaging technique used by dermatologists and is more efficient for both diagnosis of Melanocytic and Non Melanocytic lesions~\cite{Gerger2006}~\cite{Guitera2009}~\cite{Haroon2017}. Furthermore, this modality can provide slices at different depths of the skin by adjusting wavelength property and focal point~\cite{Kolm2012}. In opposition to previous modalities, \ac{rcm} remains expensive, although the number of users continue to increase in the last few years~\cite{Batta2015}. In recent years, researchers start to improve portability of the \ac{rcm} devices~\cite{Freeman2018}.\par 
By contrast, a few number of papers was published on automated ways of making diagnosis with \ac{rcm} despite their promising results in clinical context with specialists. Through these works, several researches are using either spatial features based on Gray Level Co-occurrence Matrix~\cite{Wiltgen2008}~\cite{Koller2011}, or frequency features such as wavelet-base approach for instance~\cite{Wiltgen2008}~\cite{Koller2011}~\cite{Halimi2017a}.\par 
In this paper, we aim at detecting one of the four form of skin cancer~\cite{LeGal2011}, \ie Lentigo pathology. A recent research paper, reported a similar problematic while focusing on pertinence of the Wavelets for classification into Healthy and Lentigo skin pathology~\cite{Halimi2017a}. We use it as a reference technique, as it serves the same purpose with quite good results. Furthermore, we propose to do classification according to three classes: Healthy skin, Benign and Malignant Lentigo. Then, we introduce another hand-crafted descriptor, "Haralick", and an automatic one using \ac{cnn}. For fair comparison, these two methods are evaluated the same way as wavelets descriptors.\par
The next Section covers our methodology, and experiments. Then in Section III, we discuss about the results, introducing our dataset and make an analysis of the metrics. Finally, we conclude and draw some perspectives for this work.\par