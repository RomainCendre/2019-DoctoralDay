%
% ACPR 2015 template adaptation
%! TEX root = master.tex
\documentclass[10pt,twocolumn,letterpaper]{article}

%% Latex documents that need direct input
\input{latex/article/package.tex}        % contains the latex packages
\title{Reflectance Confocal Microscopy and Lentigo: Evaluation of several feature extractors for automatic detection}

\author{Romain Cendre\\
Laboratoire ImViA, EA 7535\\
Universite de Bourgogne, Dijon, France\\
{\tt\small romain.cendre@gmail.com}
\and
Alamin Mansouri\\
Laboratoire ImViA, EA 7535\\
Universite de Bourgogne, Dijon, France\\
{\tt\small alamin.mansouri@u-bourgogne.fr}
\and
Pr Jean-Luc Perrot PhD MD\\
Service de Dermatologie-Oncologie-Allergologie\\
CHU de St Etienne\\
{\tt\small j.luc.perrot@chu-st-etienne.fr}
\and
Elisa Cinotti PhD MD\\
U.O. Dermatologia\\
Dipartimento di Scienze Mediche, Chirurgiche e Neuroscienze\\
A.O.U.S. Le Scotte - Università degli Studi di Siena\\
{\tt\small elisacinotti@gmail.com}
\and
Franck Marzani\\
Laboratoire ImViA, EA 7535\\
Universite de Bourgogne, Dijon, France\\
{\tt\small franck.marzani@u-bourgogne.fr}
}
             % contains the Title and Autor info
%% Acronym definition example using glossaries package
%% \usepackage{acro} is required
%% 
%% For a powerful usage of the acro package look at http://tex.stackexchange.com/questions/135975/how-to-define-an-acronym-by-using-other-acronym-and-print-the-abbreviations-toge

\DeclareAcronym{cnn}{
  short = CNN,
  long = Convolutional Neural Network
}

\DeclareAcronym{pca}{
  short = PCA,
  long = Principal Component Analysis
}

\DeclareAcronym{rcm}{
  short = RCM,
  long = Reflectance Confocal Microscopy
}

\DeclareAcronym{roc}{
  short = ROC,
  long = Receiver Operating Characteristic
}

 

\acprfinalcopy % *** Uncomment this line for the final submission

\def\acprPaperID{****} % *** Enter the acpr Paper ID here
\def\httilde{\mbox{\tt\raisebox{-.5ex}{\symbol{126}}}}

% Pages are numbered in submission mode, and unnumbered in camera-ready
\ifacprfinal\pagestyle{empty}\fi
\bibliography{./content/bibliography.bib}

\begin{document}

\maketitle
\thispagestyle{plain}
\pagestyle{plain}

\begin{abstract}
Detection of skin cancer in time used to be a challenge over decades. In clinical circumstances, a couple of imaging techniques aims to improve early recognition of malignant pathologies. For instance, Reflectance Confocal Microscopy modality is a suitable technique for skin pathologies diseases in clinical context. This study intends to classify these images under three categories: Healthy, Benign and Malignant Lentigo. First, we reproduce a recent work based on wavelets descriptors and apply it to these three classes. Then, we introduce two other extraction ways based on Haralick textures descriptors and a Deep features extraction.
\end{abstract}

%% Incldue the content without .tex extension
\include*{content/article/introduction}
\include*{content/article/methods}
\include*{content/article/results}
\include*{content/article/conclusion}

\begin{aknowledgements}
This research was supported by the Conseil Regional de Bourgogne Franche-Comte, France and the European Regional Development Fund (ERDF).\par
We thank Dr. Perrot and Dr. Cinotti for their work on their dataset and for their permission to use it.\par
\end{aknowledgements}

{\small
\printbibliography
}
\end{document}